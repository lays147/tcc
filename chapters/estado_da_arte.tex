\section{Estado da Arte}
O mundo da Impressão 3D possui um fluxo bem definido: 

 
%%%%%%%%%%%%%%%%%%%%%%%%%%%%%%%%%%%%%%%%%%%%%%%%%%%%%%%%%%%%%%%
%
% Welcome to Overleaf --- just edit your LaTeX on the left,
% and we'll compile it for you on the right. If you give
% someone the link to this page, they can edit at the same
% time. See the help menu above for more info. Enjoy!
%
% Note: you can export the pdf to see the result at full
% resolution.
%
%%%%%%%%%%%%%%%%%%%%%%%%%%%%%%%%%%%%%%%%%%%%%%%%%%%%%%%%%%%%%%%
% A bottom-up chart of a TeX workflow
% Author: Stefan Kottwitz
% https://www.packtpub.com/hardware-and-creative/latex-cookbook
\begin{comment}
:Title: A bottom-up chart of a TeX workflow
:Tags: Diagrams;Flowcharts;Smartdiagram;Cookbook
:Author: Stefan Kottwitz
:Slug: smart-priority

A priority chart showing a common TeX workflow
from bottom to top, using the smartdiagram package.
\end{comment}
\begin{adjustbox}{width=\linewidth}
\smartdiagram[flow diagram:horizontal]{Modelagem,
  Fatiamento, Impressão, Acabamento}
\end{adjustbox}



Dado a sua natureza Open Source, que se iniciou nos anos 2000, quando as primeiras patentes relacionadas a essa tecnologia tiveram seu vencimento, é natural que os softwares envolvidos neste mundo também tenham seu surgimento ligado a iniciativa Open Source.
Entretanto, alguns dos softwares existentes, não atendem a todas as necessidades que um usuário de uma Impressora 3D venha a ter, principalmente no que se diz respeito aos Printer Hosts.
Como definido, Printer Host é um software controlador, onde via uma conexão serial, é possível enviar e receber comandos de uma Impressora 3D. Porém, esta não é sua principal utilidade, e encontramos Printer Hosts com algumas features que tem por objetivo dar mais do que o controle da Impressora 3D. 

Alguns dos Printer Hosts existentes, que são Open Source, são:
\begin{itemize}
\item [Octoprint]
O Octoprint foi criado para ser um Host remoto, ou seja, você o instala em computador remoto, normalmente um RaspberryPi, e conecta a Impressora 3D a esse controlador, com o Octoprint instalado, você consegue via web acessar o software e fazer o controle de sua impressora. O mesmo é escrito em JavaScript, e além das features de gerenciamento de impressoras, ele também realiza o fatiamento, monitoração de impressão via vídeo, ...
TODO: Adicionar Prints do Octoprint

\item [Repetier Host]

\item [MatterControl]

\end{itemize}

\subsection{Código Livre x Código Proprietário}
\subsection{Exemplos de soluções existentes}
\subsubsection{Repetier Host}
\subsubsection{OctoPrint}
\subsubsection{MatterControl}
\subsection{Desafios do Mercado}
