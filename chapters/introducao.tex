\section{Introdução}
A impressão 3D é uma tecnologia que teve seus primórdios industriais nos anos de 1970,
com as primeiras tentativas de fabricação de peças de forma aditiva (Citação Patola).
A partir dos anos 80, surgiram as primeiras patentes, e no inicío dos anos 2000,
a tecnologia obteve um crescimento exponencial com a queda das patentes e o projeto RepRap.

A tecnologia de impressão 3D e sua estrutura base consistem de hardware e software,
como todo sistema eletrônico. E sua estrutura pode ser definida pela árvore abaixo:
\newline

\begin{comment}
:Title: Hierarchical diagram
:Tags: Coordinate calculations;Forest;Diagrams
:Author: cfr
:Slug: hierarchical-diagram

The following diagram uses the forest package to create the diagram as a
tree. Shading gives a little depth to the nodes, the shadows library enhances
this effect. Two phantom children are used to help aligning the final nodes
of the tree and the connecting lines to the first of these are added after
the tree is complete, since this node has four parents.

This example was written by cfr answering a question on TeX.SE.
\end{comment}
\usetikzlibrary{arrows.meta, shapes.geometric, calc, shadows}

\colorlet{mygreen}{green!75!black}
\colorlet{col1in}{red!30}
\colorlet{col1out}{red!40}
\colorlet{col2in}{mygreen!40}
\colorlet{col2out}{mygreen!50}
\colorlet{col3in}{blue!30}
\colorlet{col3out}{blue!40}
\colorlet{col4in}{mygreen!20}
\colorlet{col4out}{mygreen!30}
\colorlet{col5in}{blue!10}
\colorlet{col5out}{blue!20}
\colorlet{col6in}{blue!20}
\colorlet{col6out}{blue!30}
\colorlet{col7out}{orange}
\colorlet{col7in}{orange!50}
\colorlet{col8out}{orange!40}
\colorlet{col8in}{orange!20}
\colorlet{linecol}{blue!60}

\pgfkeys{/forest,
  rect/.append style   = {rectangle, rounded corners = 2pt,
                         inner color = col6in, outer color = col6out},
  ellip/.append style  = {ellipse, inner color = col5in,
                          outer color = col5out},
  orect/.append style  = {rect, font = \sffamily\bfseries\LARGE,
                         text width = 325pt, text centered,
                         minimum height = 10pt, outer color = col7out,
                         inner color=col7in},
  oellip/.append style = {ellip, inner color = col8in, outer color = col8out,
                          font = \sffamily\bfseries\large, text centered}}
\begin{adjustbox}{width=\linewidth}
\begin{forest}
  for tree={
      font=\sffamily\bfseries,
      line width=1pt,
      draw=linecol,
      ellip,
      align=center,
      child anchor=north,
      parent anchor=south,
      drop shadow,
      l sep+=12.5pt,
      edge path={
        \noexpand\path[color=linecol, rounded corners=5pt,
          >={Stealth[length=10pt]}, line width=1pt, ->, \forestoption{edge}]
          (!u.parent anchor) -- +(0,-5pt) -|
          (.child anchor)\forestoption{edge label};
        },
      where level={3}{tier=tier3}{},
      where level={0}{l sep-=15pt}{},
      where level={1}{
        if n={1}{
          edge path={
            \noexpand\path[color=linecol, rounded corners=5pt,
              >={Stealth[length=10pt]}, line width=1pt, ->,
              \forestoption{edge}]
              (!u.west) -| (.child anchor)\forestoption{edge label};
            },
        }{
          edge path={
            \noexpand\path[color=linecol, rounded corners=5pt,
              >={Stealth[length=10pt]}, line width=1pt, ->,
              \forestoption{edge}]
              (!u.east) -| (.child anchor)\forestoption{edge label};
            },
        }
      }{},
  }
  [Impressão 3D, inner color=col1in, outer color=col1out
    [Hardware, inner color=col2in, outer color=col2out
      [Firmware, inner color=col4in, outer color=col4out]
    ]
    [Software, inner color=col3in, outer color=col3out
      [Printer Host
        [Software Mediador\\Controlador, rect, name=sse1
        ]
      ]
      [Fatiadores
        [Software que converte\\ modelos em 3d\\para arquivos GCode, rect, name=sse2
        ]
      ]
      [3D Modeling
        [Softwares de Apoio a \\modelagem 3D, rect, name=sse3
        ]
      ]
    ]
  ]
  \begin{scope}[color = linecol, rounded corners = 5pt,
    >={Stealth[length=10pt]}, line width=1pt, ->]
    \coordinate (c1) at ($(sse1.south)!2/5!(sse2.south)$);
    %\coordinate (c2) at ($(sse3.south)!2/5!(sse4.south)$);
  \end{scope}
\end{forest}
\end{adjustbox}


\newpage

\section{Prototipagem Rápida}
\citet{rapidproto} afirma que um protótipo é "Uma aproximação de um produto(ou sistema)
ou de seus componentes em alguma forma para um propósito definido em sua implementação.".
Ou seja, podemos usar um protótipo para avaliar provas de conceitos, melhorar produtos
já existentes ou experimentar novas ideias relativas ao desenvolvimento de um produto.
\citet{rapidproto} também acrescenta que "Nada é mais claro para a explicação ou comunicação
de uma ideia do que um protótipo físico onde o público-alvo pode ter uma experiência visual
e táctil de um produto."

A Prototipagem Rápida permite que o desenvolvimento de um produto, de seu rascunho
a um produto final seja agilizada usando tecnologias baseados em CAD(Computer Aided Design).
O uso de tecnologias CAD permitem que os protótipos virtuais de um produto
\citet{rapidproto} "sejam estressados, testados e analisados como se eles fossem protótipos físicos".

As aplicações da Prototipagem Rápida se encontram em diversos mercados. Como o mercado
automotivo, onde empresas usam desta tecnologia para otimizar partes de um carro, ou na medicina,
onde já é possível produzir próteses de titâneo que se encaixam perfeitamente ao rosto de um paciente.

O uso de Prototipagem Rápida possui várias vantagens para o usuário, como redução do desperdício
de material e de mão de obra. Além de agilizar o processo de validação do conceito proposto.

\subsection{Histórico}
\subsection{Principais Tecnologias}
\subsection{Impressoras RepRap x Comerciais}
\subsection{101 Impressão 3D}
